\documentclass[a4paper]{article}
\usepackage{times}
\usepackage[utf8]{inputenc}
\usepackage[croatian]{babel}
\usepackage[T1]{fontenc}
\usepackage{fullpage}
\usepackage{color}
\usepackage{amsmath}
\usepackage {amsfonts}

\begin{document}
 
\section{Sažetak}
Klasificiranje mentalnog zdravlja autora analiziranjem teksta koristeći SVM i ostale
tehnike strojnog učenja. 

Dijagnosticiranje mentalne bolesti je zahtjevan problem.

Specifičan način funkcioniranja ljudi s mentalnim poremećajima (schizofrenija, bipolarni) očituje
se i u njihovom načinu korištenja jezika.

Zbog toga je moguće da se analizom tekstova dobije podatak u kom skrenu krenuti.

Nešto o svm-u  -- što je i zašto je dobar za tekst

Mislimo raditi na pravim bolesnicima ako uspijemo pribaviti tekstove, a ako ne onda 
koristiti tekstove autora (pisaca, novinara, blogova).

Java, weka, libsvm
\end{document}